\usepackage{graphicx}

\usepackage{geometry}

%\usepackage{scrextend}

\usepackage{titling}

\usepackage{tipa}

\usepackage{soul}

\usepackage{multido}

\usepackage{fontspec}
\usepackage{microtype}
%\usepackage[tracking=smallcaps]{microtype}

\usepackage{anyfontsize}

\usepackage{luacode}
\usepackage{luaotfload}

\begin{luacode}

local FONT = fontloader.open("./fonts/ECGaramond08-Regular.ttf")

function getglyph2 ( f, a )
 local u = unicode.utf8.byte( a )
 if f then
  if f.glyphcnt > 0 then
   for i=f.glyphmin,f.glyphmax do
    local g = f.glyphs[i]
    if g then
     if g.unicode == u then
      return( g )
     end
    end
   end
  end
 end
return(nil)
end

function getglyph ( a )
 local u = unicode.utf8.byte( a )
 local f = fonts.hashes.identifiers[font.current()]
 local font = fontloader.open(f.filename)
 if font then
  if font.glyphcnt > 0 then
   for i=font.glyphmin,font.glyphmax do
    local g = font.glyphs[i]
    if g then
     if g.unicode == u then
      return( g )
     end
    end
   end
  end
  fontloader.close(font)	
 end
return( nil )
end

function getkern ( a )
 local g = getglyph2( FONT, a )
 if g then
  local k = g.boundingbox[3] - g.width
   if k > 0 then
    return( k )
   end
 end
return( 0 )
end

function gkerntable( k, a )
local fac = k + a
 return( { ["dagger.sups"] = fac,
  	    ["daggerdbl.sups"] = fac,
    	    ["asterisk.sups"] = fac,
            ["one.sups"] = fac,
            ["two.sups"] = fac,
	    ["three.sups"] = fac,
	    ["four.sups"] = fac,
	    ["five.sups"] = fac,
   	    ["six.sups"] = fac,
    	    ["seven.sups"] = fac,
    	    ["eight.sups"] = fac,
            ["nine.sups"] = fac,
            ["zero.sups"] = fac,
            ["quoteright"] = fac,
            ["parenleft.sups"] = fac,
            ["parenright"] = fac,
            ["a.sups"] = fac,
	    ["b.sups"] = fac,
	    ["c.sups"] = fac,
	    ["d.sups"] = fac,
   	    ["e.sups"] = fac,
    	    ["f.sups"] = fac,
    	    ["g.sups"] = fac,
            ["h.sups"] = fac,
            ["i.sups"] = fac,
            ["j.sups"] = fac,
	    ["k.sups"] = fac,
	    ["l.sups"] = fac,
	    ["m.sups"] = fac,
   	    ["n.sups"] = fac,
    	    ["o.sups"] = fac,
    	    ["p.sups"] = fac,
            ["q.sups"] = fac,
            ["r.sups"] = fac,
            ["s.sups"] = fac,
	    ["t.sups"] = fac,
            ["u.sups"] = fac,
            ["v.sups"] = fac,            
	    ["w.sups"] = fac,
	    ["x.sups"] = fac,
   	    ["y.sups"] = fac,
    	    ["z.sups"] = fac,
    	    ["A.sups"] = fac,
            ["B.sups"] = fac,
            ["C.sups"] = fac,
            ["D.sups"] = fac,
	    ["E.sups"] = fac,
	    ["F.sups"] = fac,
	    ["G.sups"] = fac,
   	    ["H.sups"] = fac,
    	    ["I.sups"] = fac,
    	    ["J.sups"] = fac,
            ["K.sups"] = fac,
            ["L.sups"] = fac,
            ["M.sups"] = fac,
	    ["N.sups"] = fac,
	    ["O.sups"] = fac,
	    ["P.sups"] = fac,
   	    ["Q.sups"] = fac,
    	    ["R.sups"] = fac,
    	    ["S.sups"] = fac,
            ["T.sups"] = fac,
            ["U.sups"] = fac,
            ["V.sups"] = fac,
	    ["W.sups"] = fac,
	    ["X.sups"] = fac,
	    ["Y.sups"] = fac,
   	    ["Z.sups"] = fac } )
end	    

local fkern = getkern( "f" )
local Fkern = getkern( "F" )
local Nkern = getkern( "N" )
local Jkern = getkern( "J" )
local Tkern = getkern( "T" )
local Vkern = getkern( "V" )
local vkern = getkern( "v" )
local Ykern = getkern( "Y" )
local ykern = getkern( "y" )
local Ukern = getkern( "U" )


local skerns = { ["f"] = gkerntable(fkern, 0),
            ["F"] = gkerntable(Fkern, 0),
            ["V"] = gkerntable(Vkern, 0),
            ["v"] = gkerntable(vkern, 0),            
            ["N"] = gkerntable(Nkern, 0),            
            ["J"] = gkerntable(Jkern, 0),            
	    ["T"] = gkerntable(Tkern, 50),
	    ["Y"] = gkerntable(Ykern, 0),
            ["U"] = gkerntable(Ukern, 0),
            ["y"] = gkerntable(ykern, 0),
 }

fonts.handlers.otf.addfeature {
 name = "superkerns",
 type = "kern",
 data = skerns
}

function putchar ( a )
 local c = luaotfload.aux.slot_of_name(font.current(), a, false)
 if c and type(c) == "number" then
  tex.sprint( string.format( [[\char"%X]], c ))
 else
  tex.error("No character [ " .. a .. " ] in current font")
 end
end

function getutfname ( a )
 local g = getglyph( a )
 if g then
      return( g.name )
 end
return( nil )
end

function tosups ( a )
 for i=1, unicode.utf8.len(a) do
  local p = unicode.utf8.sub( a, i, i )
  if p == " " then
   tex.sprint( "\\ " )
  else
   local n = getutfname( p )
   if n and type(n) == "string" then
    putchar( n .. ".sups" )
   end
  end
 end
end

\end{luacode}

\def\realsuper#1{{\directlua{tosups("#1")}}}

\setmainfont[Ligatures=TeX,
Path=./fonts/,
UprightFeatures={RawFeature=+superkerns},
BoldFont=EBGaramond-SemiBold.ttf,
ItalicFont=EBGaramond08-Italic.ttf,
BoldItalicFont=EBGaramond-SemiBoldItalic.ttf,
SlantedFont=EBGaramond08-Italic.ttf,
BoldSlantedFont=EBGaramond-SemiBoldItalic.ttf,
UprightFont=ECGaramond08-Regular.ttf]{ECGaramond08-Regular}

\newenvironment{allsc}{%
  \normalfont%
  \addfontfeatures{RawFeature=+u2sc}%
  \scshape%
}{}

\linespread{1}

\newenvironment{vplace}{
  \vspace*{\stretch{1}}
}{
  \vspace{\stretch{1}}
}

%% u2sc is only in customized ECGaramond08-Regular
\def\scupcase{\normalfont\addfontfeatures{RawFeature=+u2sc}\scshape }
\def\altnumber{\addfontfeatures{Numbers={Proportional,Lining}} }

\newlength{\THINKERN}
\setlength{\THINKERN}{0.16667em}
\DeclareRobustCommand{\thinskip}{\hskip 0.16667em\relax}
\DeclareRobustCommand{\thinkern}{\kern 0.16667em\relax}

\DeclareRobustCommand{\tinyskip}{\hskip 0.1em\relax}
\DeclareRobustCommand{\tinykern}{\kern 0.1em\relax}

%% follow dash conventions, ie emdash left at right margin
%% when breaking
%% ellipsis does not do this.  Should it?
\def\emdash{\unskip—\allowbreak\ignorespaces}

\def\emdashSpaceAfter{\unskip{}—\space}

\def\emdashEnd{\unskip{}—}
\def\emdashStart{\leavevmode{}—\ignorespaces}
\def\emdashSpaceBeforeAfter{\unskip\quad{}—\quad\ignorespaces}

\def\emdashDialogBeg{—\ignorespaces}
\def\emdashDialogEnd{\unskip—}

\DeclareRobustCommand{\negtinykern}{\kern -0.3em\relax}
\def\twoemdash{\unskip—\negtinykern —\space}

\def\LeftCurlyQuote{“\tinykern\ignorespaces}
\def\RightCurlyQuote{\unskip\unkern\tinykern{}”}

\def\LeftCurlyQuoteS{‘\tinykern\ignorespaces}
\def\RightCurlyQuoteS{\unskip\unkern\tinykern{}’}

\DeclareRobustCommand{\ellipskip}{\hskip 0.1em\relax}
\DeclareRobustCommand{\ellipkern}{\kern 0.1em\relax}

\def\ellip{\ellipskip .\ellipkern .\ellipkern .}
\def\ellipHold{\ellipkern .\ellipkern .\ellipkern .}

\def\ellipsisSpaceAfter{\unskip\ellipHold\space}

\def\ellipsis{\unskip\ellip\ellipskip\ignorespaces}

\def\ellipsisendquote{\unskip\ellipHold\thinkern ”}

\def\ellipsisComma{\unskip\ellipHold\ellipkern ,}
\def\ellipsisQuestion{\unskip\ellipHold\ellipkern ?}
\def\ellipsisPeriod{\unskip\ellipHold\ellipkern .}
\def\ellipsisExclamation{\unskip\ellipHold\ellipkern !}

\def\ellipsisEnd{\unskip\ellipHold\ignorespaces}

\def\ellipsisSpaceBeforeAfter{\unskip\quad .\ellipkern .\ellipkern .\quad\ignorespaces}

\def\ellipsisSpaceBefore{\unskip\quad .\ellipkern .\ellipkern .\ellipkern}

\def\ellipsisNoBreak{\unskip\ellipHold\ellipkern\ignorespaces}
\def\ellipsisOnlyLine{\ellip}

\def\ellipsisStart{\leavevmode{}.\ellipkern .\ellipkern .\thinskip\ignorespaces}
\def\ellipsisStartSpace{\leavevmode{}.\ellipkern .\ellipkern .\thinskip\ignorespaces\ }

\def\ellipsisBracket{\unskip\thinkern[\ellipkern .\ellipkern .\ellipkern .\ellipkern]\thinkern\ignorespaces}
\def\ellipsisBracketEnd{\unskip\thinkern[\ellipkern .\ellipkern .\ellipkern .\ellipkern]\ignorespaces}

\def\ellipsisBracketMid{\unskip\thinkern{}[\ellipkern .\ellipkern .\ellipkern .\ellipkern]\thinkern\ignorespaces}

\def\zeroJoinAllowBreak{\unskip\allowbreak\ignorespaces}
\def\zeroJoinNoBreak{\unskip\ignorespaces}
\def\dashJoinAllowBreak{\unskip -\allowbreak\ignorespaces}

\def\rStartQuote{”\ignorespaces }
\def\rEndQuote{\unskip\tinykern “}

\def\makedots[#1]{\unskip\thinkern .\ellipkern .\multido{}{\numexpr #1 - 2\relax}{\ellipskip .}\thinskip\ignorespaces}

\def\makedotsIsland[#1]{\unskip\thinskip .\ellipkern .\multido{}{\numexpr #1 - 2\relax}{\ellipskip .}\thinskip\ignorespaces}

\def\makedotsEnd[#1]{\unskip\thinkern .\ellipkern .\multido{}{\numexpr #1 - 2\relax}{\ellipskip .}}

\def\makedotsStart[#1]{\leavevmode\unskip .\ellipkern .\multido{}{\numexpr #1 - 2\relax}{\ellipskip .}\ignorespaces\ }

\def\sixDots{\makedots[6]}
\def\sevenDots{\makedots[7]}
\def\nineDots{\makedots[9]}

\def\fiveDots{\makedots[5]}
\def\elevenDots{\makedots[11]}
\def\elevenIslandDots{\makedotsIsland[11]}

\def\fiveEndDots{\makedotsEnd[5]}
\def\sevenEndDots{\makedotsEnd[7]}
\def\fifteenEndDots{\makedotsEnd[15]}

\def\twelveDots{\makedots[12]}
\def\fifteenDots{\makedots[15]}
\def\eighteenDots{\makedots[18]}
\def\twentyoneDots{\makedots[21]}
\def\fortytwoDots{\makedots[42]}
\def\seventyfiveDots{\makedots[75]}

\def\sixIslandDots{\makedotsIsland[6]}
\def\twelveIslandDots{\makedotsIsland[12]}

\def\sixEndDots{\makedotsEnd[6]}

\def\nineEndDots{\makedotsEnd[9]}
\def\nineStartDots{\makedotsStart[9]}

\def\twelveEndDots{\makedotsEnd[12]}

\def\eighteenEndDots{\makedotsEnd[18]}

\def\twentyoneEndDots{\makedotsEnd[21]}

\def\fortytwoEndDots{\makedotsEnd[42]}


\def\hanging{\hangindent=2em\hangafter=1}

\newenvironment{disableindent}{%
\setlength{\parindent}{0pt}%
\setlength{\parskip}{\baselineskip}%
\vspace{-\baselineskip}%
}{%
}

\newenvironment{disableIndentStretch}{%
\setlength{\parindent}{0pt}%
\setlength{\parskip}{1\baselineskip plus 2pt minus 2pt}%
}{%
\vspace{1\baselineskip plus 2pt minus 2pt}
}


\newenvironment{doublespacedpara}{%
\setlength{\parskip}{\baselineskip}%
}{%
}

\newenvironment{disableindentonly}{%
\setlength{\parindent}{0pt}%
}{%
}


\newenvironment{disableindentb}{%
\setlength{\parindent}{0pt}%
}{%
}


%% this isn't really a true monofont, some characters,
%% such as [ and ] and ( and ), are off center
\newfontfamily{\teletypefont}{LMTTMono8-Regular.ttf}[Path=./fonts/]
\newfontfamily{\monofont}{LMTTMono8-Regular.ttf}[Path=./fonts/]
\newfontfamily{\cryptofont}{monocrypto.ttf}[Path=./fonts/]

\newenvironment{teletype}{%
\setlength{\parindent}{0pt}\setlength{\parskip}{0em}
\teletypefont
\fontsize{8pt}{9pt}\selectfont
}{%
\normalsize
}

\newenvironment{teletypeB}{%
\setlength{\parindent}{0pt}\setlength{\parskip}{0em}
\teletypefont
\fontsize{8pt}{9pt}\selectfont
}{%
\normalsize
}

\newcommand{\blankpage}{
\thispagestyle{empty}
\mbox{}
\clearpage
}

\newcommand{\blankpageB}{
\thispagestyle{plain}
\mbox{}
\clearpage
}

\def\endpadpage{\clearpage{\thispagestyle{empty}\cleardoublepage\blankpage\blankpage}}


\def\doCATCHLINE{%
\begin{vplace}%
\begin{center}%
\begin{minipage}{.7\textwidth}%
\rule{\textwidth}{.5\baselineskip}
\par
\vspace*{\baselineskip}
\begingroup
\setlength{\parindent}{0pt}
\setlength{\parskip}{\baselineskip}
\vskip-\baselineskip
\large
\hfill Well, is it?\hfill\null

\endgroup
\vspace*{\baselineskip}
\par
\rule{\textwidth}{.5\baselineskip}
\end{minipage}
\end{center}
\end{vplace}
}


\def\REALdoCOPYING{
{\centering\rule{\textwidth}{.5\baselineskip}\par}
\vspace{.5\baselineskip}
{\large\centering
EXTENDED COPYING INFO HERE\par}
\vspace{.5\baselineskip}
{\centering\rule{\textwidth}{.5\baselineskip}\par}
\vfill
}

%%% underline thickness
\setul{1pt}{.75pt}

\def\FFbox#1{{\setlength{\fboxsep}{1pt}%
\setlength{\fboxrule}{1pt}%
\fbox{#1}}}

\usepackage{needspace}

\newcounter{ChapterCounter}
\setcounter{ChapterCounter}{1}
\def\chapterBreakN{\cleardoubleoddpage{\centering {\Huge \arabic{ChapterCounter}} \\}\vspace{7\baselineskip}\stepcounter{ChapterCounter}}


\newlength{\LHangOn}

\def\hangOn#1#2{%
\settowidth{\LHangOn}{#2}
\parshape 2 0pt \textwidth
\dimexpr\LHangOn+#1\relax \dimexpr\textwidth-\LHangOn-#1\relax
\noindent
\mbox{#2}\kern#1\relax\ignorespaces}

\emergencystretch 3em

%\raggedbottom
\flushbottom

\newlength{\DOsectHEIGHT}
\newlength{\DOsectMinHEIGHT}
\newlength{\DOsectMaxHEIGHT}
\newlength{\DOsectAVAIL}

\def\DOsectionPUT#1#2#3{%
\ifdim #1>0pt
\vspace{#1}
\fi
\hfill #3 \hfill\null
\ifdim #2>0pt
\vspace{#2}
\fi}

\def\DOsectionBREAK#1#2#3{%
\begingroup
\settototalheight{\DOsectHEIGHT}{#3}
\setlength{\DOsectMinHEIGHT}{\dimexpr\DOsectHEIGHT+#1}
\setlength{\DOsectMaxHEIGHT}{\dimexpr\DOsectHEIGHT+#1+#2+2\baselineskip}
\setlength{\DOsectAVAIL}{\dimexpr\pagegoal-\pagetotal-\baselineskip}
\ifdim \DOsectAVAIL=\DOsectMaxHEIGHT
    \DOsectionPUT{#1}{#2}{#3}
\else
    \ifdim \DOsectAVAIL>\DOsectMaxHEIGHT
        \DOsectionPUT{#1}{#2}{#3}
    \else
        \ifdim \DOsectMinHEIGHT=\DOsectAVAIL
            \DOsectionPUT{#1}{0pt}{#3}
            \clearpage
         \else
            \ifdim \DOsectMinHEIGHT<\DOsectAVAIL
                \DOsectionPUT{#1}{0pt}{#3}
                \clearpage
             \else      
                \clearpage
                \DOsectionPUT{0pt}{#2}{#3}                  
             \fi
         \fi
     \fi
\fi
\endgroup}

\usepackage{calc}

%\widowpenalty=0
\widowpenalty=10000
%\widowpenalty=150
%\widowpenalty=0
%\displaywidowpenalty=0
\displaywidowpenalty=10000
\clubpenalty=0
\brokenpenalty=10000

\newlength{\BASIClines}
\setlength{\BASIClines}{37\baselineskip}

\newlength{\WEBleft}
\setlength{\WEBleft}{.75in}

\newlength{\WEBright}
\setlength{\WEBright}{.75in}

\newlength{\PRINTleft}
\setlength{\PRINTleft}{.875in}

\newlength{\PRINTright}
\setlength{\PRINTright}{.625in}

\newlength{\BOTTOM}
\setlength{\BOTTOM}{.975in}

\newlength{\TOP}
\setlength{\TOP}{.55in}

\newlength{\FOOTS}
\setlength{\FOOTS}{.40in}

\ifdefined\isPRINT
\else
\let\isPRINT=y
\fi

\ifx\isPRINT y
    \geometry{footskip=\FOOTS,
    left=\PRINTleft,
    right=\PRINTright,
    bottom=\BOTTOM,
    textheight=\BASIClines,
    heightrounded=true}
\else
    \geometry{footskip=\FOOTS,
    left=\WEBleft,
    right=\WEBright,
    bottom=\BOTTOM,
    textheight=\BASIClines,
        heightrounded=true,
        showframe=true}
\fi

\ifx\isPRINT y
\else
    \usepackage{pdfpages}
\fi

\def\MAYBEdoENDPAGE{%
\ifx\isPRINT y
    \endpadpage
\fi
}

\def\openPRINTorWEB{%
\ifx\isPRINT y
    \blankpage\cleardoubleoddpage
\else
    \includepdf{cover.pdf}
    \clearpage
    \doCATCHLINE
    \clearpage
\fi
}

\setlength{\parskip}{0pt plus 0pt minus 0pt}


\ifdefined\doSHOWFRAME
\else
\let\doSHOWFRAME=n
\fi

\ifx\doSHOWFRAME y
    \usepackage{luatexbase}
    \directlua{dofile("widow-assist.lua")}
    \usepackage{showframe}
\fi

\ifdefined\doCOPYING
\else
\let\doCOPYING=n
\fi

\def\COPYING{%
\ifx\doCOPYING y
    \REALdoCOPYING
\fi
}

